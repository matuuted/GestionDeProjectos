\documentclass[
11pt, % The default document font size, options: 10pt, 11pt, 12pt
%codirector, % Uncomment to add a codirector to the title page
]{charter} 


% El títulos de la memoria, se usa en la carátula y se puede usar el cualquier lugar del documento con el comando \ttitle
\titulo{Dispensador inteligente de comida para gatos} 

% Nombre del posgrado, se usa en la carátula y se puede usar el cualquier lugar del documento con el comando \degreename
\posgrado{Carrera de Especialización en Sistemas Embebidos} 

% Tu nombre, se puede usar el cualquier lugar del documento con el comando \authorname
% IMPORTANTE: no omitir titulaciones ni tildación en los nombres, también se recomienda escribir los nombres completos (tal cual los tienen en su documento)
\autor{Ing. Durante Matías Nahuel}

% El nombre del director y co-director, se puede usar el cualquier lugar del documento con el comando \supname y \cosupname y \pertesupname y \pertecosupname
\director{Ing. Bualó Santiago}
\pertenenciaDirector{FIUBA} 
\codirector{} % para que aparezca en la portada se debe descomentar la opción codirector en los parámetros de documentclass
\pertenenciaCoDirector{FIUBA}

% Nombre del cliente, quien va a aprobar los resultados del proyecto, se puede usar con el comando \clientename y \empclientename
\cliente{Nombre del cliente}
\empresaCliente{Empresa del cliente}
 
\fechaINICIO{21 de octubre de 2025}		%Fecha de inicio de la cursada de GdP \fechaInicioName
\fechaFINALPlan{14 de diciembre de 2025} 	%Fecha de final de cursada de GdP
\fechaFINALTrabajo{20 de julio de 2026}	%Fecha de defensa pública del trabajo final

\newcommand{\horaTrabajo}{620}

\begin{document}

\maketitle
\thispagestyle{empty}
\pagebreak


\thispagestyle{empty}
{\setlength{\parskip}{0pt}
\tableofcontents{}
}
\pagebreak


\section*{Registros de cambios}
\label{sec:registro}


\begin{table}[ht]
\label{tab:registro}
\centering
\begin{tabularx}{\linewidth}{@{}|c|X|c|@{}}
\hline
\rowcolor[HTML]{C0C0C0} 
Revisión & \multicolumn{1}{c|}{\cellcolor[HTML]{C0C0C0}Detalles de los cambios realizados} & Fecha      \\ \hline
0      & Creación del documento                                 &\fechaInicioName \\ \hline
1      & Primera entrega               							& {04} de {noviembre} de 2025 \\ \hline
2      & Segunda entrega                                        & {11} de {noviembre} de 2025 \\ \hline
3      & Tercera entrega                 						& {18} de {noviembre} de 2025 \\ \hline
4      & Cuarta entrega	                                	 	& {25} de {noviembre} de 2025 \\ \hline

% Si hay más correcciones pasada la versión 4 también se deben especificar acá

\end{tabularx}
\end{table}

\pagebreak



\section*{Acta de constitución del proyecto}
\label{sec:acta}

\begin{flushright}
Buenos Aires, \fechaInicioName
\end{flushright}

\vspace{2cm}

Por medio de la presente se acuerda con el \authorname\hspace{1px} que su Trabajo Final de la \degreename\hspace{1px} se titulará ``\ttitle'' y consistirá en la implementación de un prototipo de un sistema de dispensador inteligante de comida para gatos, capaz de automatizar la entrega de alimento seco en horarios programados, monitorear el consumo y reportar la información a una plataforma web, y tendrá un presupuesto preliminar estimado de {\horaTrabajo} horas y un costo estimado de 9837 USD, con fecha de inicio el \fechaInicioName\hspace{1px} y fecha de presentación pública el \fechaFinalName.

Se adjunta a esta acta la planificación inicial.

\vfill

\begin{table}[ht]
\centering
\begin{tabular}{ccc}
\begin{tabular}[c]{@{}c@{}}Dr. Ing. Ariel Lutenberg \\ Director posgrado FIUBA\end{tabular} &  \vspace{2.5cm} \\ 
\multicolumn{3}{c}{\begin{tabular}[c]{@{}c@{}} \supname \\ Director del Trabajo Final\end{tabular}} \vspace{2.5cm} \\
\end{tabular}
\end{table}




\section{1. Descripción técnica-conceptual del proyecto a realizar}
\label{sec:descripcion}

El presente trabajo práctico busca implementar un sistema de alimentación inteligente para mascotas, capaz de administrar y controlar de manera automática la entrega de alimento, supervisando su consumo y registrando la información correspondiente. El sistema tiene como finalidad facilitar el cuidado de los gatos en el hogar, asegurando que reciban la cantidad adecuada de comida en los horarios establecidos, incluso en ausencia de los dueños.

Se ha observado que una de las problemáticas más comunes entre los propietarios de mascotas es la dificultad para garantizar una alimentación constante y controlada, especialmente cuando no se encuentran en el hogar debido a largas jornadas laborales o deben ausentarse por períodos prolongados. En muchos casos, se depende de terceros o de dispensadores automáticos simples que no brindan información sobre si el animal efectivamente se alimentó, ni permiten llevar un registro del consumo diario. El objetivo principal de este proyecto es dar una solución tecnológica a esta necesidad, aplicando principios de automatización y monitoreo inteligente, a fin de mejorar la calidad de vida tanto de las mascotas como de sus dueños.

El sistema a desarrollar se basará en un conjunto de sensores que permitirán controlar la cantidad de alimento dispensado y verificar su consumo, junto con un mecanismo automatizado que permitirá dosificar las raciones según los horarios programados. Además, el sistema incluirá un registro continuo de datos que posibilitará identificar patrones y comportamientos alimentarios de la mascota, aportando información valiosa para la evaluación y mejora del plan de alimentación junto a profesionales del área.
El dispositivo tendrá la capacidad de funcionar de manera autónoma, controlando la entrega del alimento, detectando si la ración fue consumida, y comunicándose con una interfaz externa, desde la cual el usuario podrá visualizar reportes, modificar horarios o recibir alertas cuando el depósito de alimento esté próximo a vaciarse.

Actualmente, existen en el mercado diversos dispositivos que permiten automatizar la entrega de comida para mascotas, pero la mayoría presenta limitaciones en cuanto al monitoreo del consumo o la disponibilidad de información remota. En la mayoría de los casos, el usuario no puede saber si el animal efectivamente comió o si se produjo alguna falla en la dispensación.
La propuesta de este proyecto busca superar esas limitaciones, ofreciendo una solución que combina la automatización, el sensado, el control inteligente y la conectividad. De esta forma, no solo se garantiza la alimentación adecuada del animal, sino que además se obtiene información confiable y útil para su seguimiento.

La iniciativa también pretende sentar las bases de un nuevo enfoque en el cuidado automatizado de mascotas, incorporando tecnologías que hasta el momento se aplicaban principalmente en entornos industriales o de domótica. Así, se impulsa un paradigma en el que la automatización del hogar se extiende al bienestar animal, contribuyendo a una gestión más eficiente del tiempo y a una mejor calidad de vida en los hogares.
Con esta nueva gestión automatizada de la alimentación, se podrán mejorar los siguientes aspectos:

\begin{itemize}
\item Mantener un control preciso de las raciones diarias y del consumo efectivo de la mascota.
\item Establecer rutinas automáticas y personalizadas de alimentación.
\item Generar reportes de consumo útiles para el seguimiento nutricional.
\end{itemize}

\vspace{2.5cm}

Para llevar a cabo este proyecto, se propone un módulo integral que gestione de forma automática el almacenamiento, dispensado y monitoreo del alimento. Dicho módulo será capaz de administrar el proceso completo de alimentación, registrando la información de consumo y enviándola a una interfaz de usuario desde la cual el propietario podrá consultar el estado del sistema, modificar configuraciones o recibir alertas en caso de que el alimento disponible se encuentre próximo a agotarse.
El sistema será autónomo, de fácil instalación y adaptable a diferentes entornos domésticos, de modo que pueda operar sin depender de una ubicación fija ni de supervisión constante.

A continuación, se presenta en la Figura \ref{fig:diagBloques} un diagrama en bloques del sistema propuesto, en el cual se observan los principales módulos que lo componen:

\begin{figure}[htpb]
\centering 
\includegraphics[width=.65\textwidth]{./Figuras/diagBloques.pdf}
\caption{Diagrama en bloques del sistema.}
\label{fig:diagBloques}
\end{figure}


\section{2. Identificación y análisis de los interesados}
\label{sec:interesados}
A continuación se listan todas las partes involucradas en el proyecto:

\begin{table}[ht]
\begin{tabularx}{\linewidth}{@{}|l|X|X|l|@{}}
\hline
\rowcolor[HTML]{C0C0C0} 
Rol           & Nombre y Apellido & Organización 	& Puesto 	\\ \hline
Responsable   & \authorname       & FIUBA        	& Alumno 	\\ \hline
Orientador    & \supname	      & \pertesupname 	& Director del Trabajo Final \\ \hline
Usuario final & Dueños de mascotas &              	&        	\\ \hline
\end{tabularx}
\caption{Identificación de los interesados} 
\label{tab:interesados}
\end{table}


\vspace{2.5cm}

\begin{itemize}
	\item \textbf{Responsable:} Ing. Durante Matías Nahuel, quien será el responsable de la planificación, diseño, desarrollo e implementación del sistema propuesto, asumiendo además todos los gastos y/o beneficios económicos al tratarse de un proyecto de carácter personal.
	\item \textbf{Orientador:}  Ing. Bualó Santiago, quien actuará como director del trabajo final, brindando acompañamiento y supervisión técnica durante el desarrollo del proyecto.
	\item \textbf{Usuario final:} dado que el proyecto se orienta a ofrecer una solución para el cuidado automatizado de mascotas, los usuarios finales serán todas aquellas personas que posean gatos y deseen optimizar su alimentación diaria mediante un sistema automático y monitoreado.
\end{itemize}

\section{3. Propósito del proyecto}
\label{sec:proposito}

El propósito de este proyecto es aplicar tecnologías de automatización y monitoreo inteligente al cuidado doméstico de mascotas, mediante el desarrollo de un sistema confiable y adaptable que administre, registre y analice la alimentación de gatos, optimizando la interacción entre el usuario y su mascota.

\section{4. Alcance del proyecto}
\label{sec:alcance}

Dentro del alcance del proyecto se contemplan las siguientes tareas principales:
\begin{itemize}
	\item El desarrollo de un dispositivo capaz de automatizar la alimentación de gatos, permitiendo dispensar el alimento mediante los siguientes modos:
		\begin{itemize}
		\item \textbf{Automático:} mediante la programación de horarios y raciones predefinidas.
		\item \textbf{Manual}: a través de un pulsador que permitirá activar el dispensado del alimento.
		\item \textbf{Remoto (opcional)}: mediante una aplicación móvil o conexión web, que posibilitará la gestión y el monitoreo del sistema a distancia.
		\end{itemize}
	\item El diseño del sistema de sensado, encargado de controlar el nivel de alimento disponible tanto en el depósito como en el plato.
	\item La implementación de un sistema de registro de datos que permita almacenar información sobre las raciones dispensadas y el consumo de alimento.
	\item El desarrollo de una interfaz externa al dispositivo, desde la cual el usuario podrá visualizar el estado del sistema, modificar los horarios de alimentación y recibir alertas ante fallas o niveles bajos de alimento.
	\item El sistema de alimentación propio del prototipo.
	
\end{itemize}

No serán parte del proyecto los siguientes:

\begin{itemize}
	\item El desarrollo de una aplicación móvil o plataforma web completa.
	\item La incorporación de identificación individual de las mascotas.
	\item La utilización de servicios de almacenamiento externo o en la nube.
	\item El desarrollo de un prototipo final destinado a su comercialización.
	\item El diseño de la estructura definitiva del dispositivo.
\end{itemize}

\section{5. Supuestos del proyecto}
\label{sec:supuestos}

Para el desarrollo del presente proyecto se supone que:

\begin{itemize}
	\item Se dispondrá de los recursos económicos para la adquisición de materiales y componentes requeridos.
	\item Los componentes electrónicos y mecánicos necesarios estarán disponibles en el mercado local.
	\item El dispositivo podrá ser probado en un entorno doméstico con suministro eléctrico.
	\item No habrá restricciones reglamentarias, técnicas o logísticas que impidan el desarrollo y la validación del prototipo.
	\item El alimento utilizado será de tipo seco, con formato y tamaño compatibles con el mecanismo de dosificación a implementar.
	
\end{itemize}

\section{6. Requerimientos}
\label{sec:requerimientos}
A continuación se enumeran los requerimientos del proyecto:
\begin{enumerate}
	\item Requerimientos funcionales:
		\begin{enumerate}
			\item El sistema debe dispensar alimento seco de manera automática en horarios programados.
			\item El sistema debe permitir el dispensado manual mediante un botón físico.
			\item El sistema debe registrar y almacenar los datos de consumo diario.
			\item El sistema debe detectar el nivel de alimento disponible en el tanque.
			\item El sistema debe medir el peso del alimento dispensado en el plato.
			\item El sistema debe emitir una alerta cuando el nivel de alimento del tanque sea bajo.
			\item El sistema debe ser capaz de operar de forma autónoma sin conexión a internet.
			\item El sistema debe permitir la visualización de reportes e historial de consumo desde una interfaz web o aplicación.
		\end{enumerate}
	\item Requerimientos de hardware:
		\begin{enumerate}
			\item El sistema deberá incorporar una placa microcontroladora de la familia STM, a fin de mantener la compatibilidad con las bibliotecas HAL (Hardware Abstraction Layer) utilizadas en el desarrollo del firmware.
			\item El sensado de peso se realizará mediante módulos HX711 y celdas de carga.
			\item El sistema deberá incorporar un módulo Wi-Fi (ESP8266 o ESP32) para la comunicación remota.
			\item El sistema deberá contar con un botón físico para el accionamiento manual.
			\item El prototipo deberá alimentarse mediante una fuente externa.
		\end{enumerate}
	\item Requerimiento de firmware:
		\begin{enumerate}
			\item El firmware deberá ser desarrollado en lenguaje C.
			\item Se deberá integrar un sistema operativo en tiempo real (FreeRTOS) para la gestión de tareas concurrentes.
			\item El sistema deberá incluir rutinas de verificación de sensores y comunicación, realizando pruebas unitarias e integrales para cada módulo.
		\end{enumerate}
	\item Requerimientos de gestión:
		\begin{enumerate}
			\item Se utilizará un repositorio público (GitHub) para el control de versiones y seguimiento del código.
			\item El tiempo total estimado para el desarrollo del prototipo será de {\horaTrabajo} horas.
			\item Se establecerán reuniones periódicas de seguimiento con el director del trabajo final para evaluar el progreso del proyecto.
		\end{enumerate}
\end{enumerate}





\section{7. Historias de usuarios (\textit{Product backlog})}
\label{sec:backlog}

En esta sección se enuncian las historias de usuario, asignándoles un puntaje según los siguientes aspectos:
\begin{enumerate}
	\item Dificultad del trabajo.
	\item Complejidad del trabajo.
	\item Riesgo asociado.
\end{enumerate}

Para asignar la ponderación, se utiliza una escala basada en la serie de Fibonacci, donde un número mayor implica un mayor costo.
El Story Point se calcula sumando las tres ponderaciones, y en caso de que el resultado no pertenezca a la serie, se redondea al número superior más cercano.

\begin{enumerate}
	\item Como dueño de mascota quiero poder programar los horarios de comida para asegurar que mi gato reciba su alimento de forma regular y sin depender de mi presencia.
		\begin{itemize}
			\item Dificultad: 5
			\item Complejidad: 5
			\item Riesgo: 8
			\item Story Point: 21
		\end{itemize}
	\item Como usuario quiero recibir una alerta cuando el tanque de alimento esté por terminarse, para poder recargarlo antes de que se quede sin comida.
		\begin{itemize}
			\item Dificultad: 8
			\item Complejidad: 5
			\item Riesgo: 3
			\item Story Point: 21
		\end{itemize}
	\item Como usuario quiero disponer en cualquier momento de la información sobre si mi gato consumió su comida, a través de la medición del peso del plato.	
		\begin{itemize}
			\item Dificultad: 8
			\item Complejidad: 8
			\item Riesgo: 5
			\item Story Point: 34
		\end{itemize}
	\item Como usuario quiero poder dispensar alimento manualmente, para darle comida a mi gato fuera de los horarios programados.
		\begin{itemize}
			\item Dificultad: 3
			\item Complejidad: 3
			\item Riesgo: 2
			\item Story Point: 8
		\end{itemize}
	\item Como usuario quiero consultar reportes sobre el consumo diario y semanal, para conocer los hábitos alimenticios de mi gato.
		\begin{itemize}
			\item Dificultad: 8
			\item Complejidad: 13
			\item Riesgo: 8
			\item Story Point: 34
		\end{itemize}	
	\item Como usuario quiero que el sistema funcione incluso sin conexión a internet, garantizando el dispensado automático aunque haya fallas en la red Wi-Fi.
		\begin{itemize}
			\item Dificultad: 5
			\item Complejidad: 8
			\item Riesgo: 8
			\item Story Point: 21
		\end{itemize}				
\end{enumerate}



\section{8. Entregables principales del proyecto}
\label{sec:entregables}

Los entregables del proyecto son:

\begin{itemize}
	\item Diagrama de bloques del sistema propuesto.
	\item Diseño esquemático del circuito electrónico.
	\item Simulación o integración del hardware en entorno de prueba.
	\item Código fuente del firmware desarrollado en lenguaje C.
	\item Repositorio con documentación técnica y control de versiones.
	\item Documentación técnica y hojas de datos de las placas y módulos utilizados.
	\item Guía de usuario.
\end{itemize}

\section{9. Desglose del trabajo en tareas}
\label{sec:wbs}

A continuación se detallan las tareas y sub-tareas, junto con su duración estimada en horas de trabajo.

\begin{enumerate}
\item Planificación y gestión del proyecto (75 h)
	\begin{enumerate}
	\item Elaboración del plan de trabajo (10 h)
	\item Definición de componentes y proveedores (15 h)
	\item Planificación del desarrollo de firmware y hardware (20 h)
	\item Redacción de informes y presentaciones (20 h)
	\item Elaboración de guía de usuario y otros documentos afines (10 h)
	\end{enumerate}
\item Investigación y análisis técnico (85 h)
	\begin{enumerate}
	\item Relevamiento de dispensadores automáticos existentes en el mercado (10 h)
	\item Selección y análisis de sensores de peso (15 h)
	\item Evaluación de opciones de conectividad (15 h)
	\item Evaluación de métodos de dispensado de alimento (10 h)
	\item Estudio de librerías de FreeRTOS y drivers compatibles (15 h)
	\item Análisis de estructura mecánica del dispensador (8 h)
	\item Investigación de protocolos Wi-Fi y BLE para comunicación del sistema (12 h)
	\end{enumerate}
\item Desarrollo de hardware (135 h)
	\begin{enumerate}
	\item Diseño esquemático del circuito (30 h)
	\item Diseño del PCB y generación de archivos de fabricación (30 h)
	\item Ensamblado y soldadura de componentes (20 h)
	\item Diseño y construcción de la estructura mecánica del dispensador (30 h)
	\item Pruebas eléctricas y de funcionamiento inicial (25 h)
	\end{enumerate}
\item Desarrollo de software (150 h)	
	\begin{enumerate}
	\item  Desarrollo de drivers específicos para cada uno de los periféricos (30 h)
	\item Implementación de la máquina de estados y funcionamiento del dispensado (25 h)
	\item Integración de FreeRTOS y gestión de tareas (30 h)
	\item Implementación de comunicación Wi-Fi con servidor (30 h)
	\item Desarrollo de registro de datos y almacenamiento local (15 h)
	\item Optimización del código y pruebas de funcionamiento (20 h)	
	\end{enumerate}	
\item Fabricación y armado del prototipo (60 h)	
	\begin{enumerate}
	\item Fabricación o montaje del PCB (10 h)
	\item Adquisición e integración de componentes (10 h)
	\item Ensamblado general del dispositivo (20 h)
	\item Pruebas funcionales y ajustes mecánicos (20 h)
	\end{enumerate}

\vspace{2.5cm}
\item Documentación (115 h)
	\begin{enumerate}
	\item Realización de informes de avance (30 h)
	\item Elaboración de la memoria técnica del proyecto (40 h)
	\item Elaboración del video demostrativo (15 h)
	\item Elaboración de la presentación final del proyecto (30 h)
	\end{enumerate}
\end{enumerate}

Cantidad total de horas: {\horaTrabajo} h

\section{10. Diagrama de Activity On Node}
\label{sec:AoN}

En la Figura \ref{fig:AoN} se puede ver el diagrama Activity on Node del presente proyecto. El diagrama incorpora la fecha de inicio y la fecha de fin del proyecto, junto con la duración de cada tarea expresada en horas. La duración total del proyecto es de 620 horas considerando todas las tareas previstas.
El camino crítico en el diagrama está representado por los bloques que tienen un recuadro rojo y flechas del mismo color, destacando la secuencia de tareas que determinan el tiempo mínimo del proyecto. La duración del camino crítico es de 290 horas.


\begin{figure}[htpb]
\centering 
\includegraphics[width=.68\textwidth]{./Figuras/OnNode.pdf}
\caption{Diagrama de \textit{Activity on Node}.}
\label{fig:AoN}
\end{figure}



\section{11. Diagrama de Gantt}
\label{sec:gantt}

En la Figura \ref{fig:diagGantt} se presenta el Diagrama de Gantt correspondiente al desarrollo del proyecto. Para su elaboración se empleó el software GanttProject, tal como se muestra en la Figura \ref{fig:Gantt}, donde se detalla el listado completo de tareas. La planificación se realizó considerando una dedicación de 2 horas diarias (equivalente a 10 horas semanales) y respetando las dependencias definidas previamente en el diagrama Activity on Node. Con esta asignación horaria, la fecha estimada de finalización es el 4 de junio de 2026, lo que brinda un margen suficiente previo a la presentación final programada para el 20 de julio de 2026.


\begin{figure}[htpb]
\centering 
\includegraphics[width=.6\textwidth]{./Figuras/Gantt-1.png}
\caption{Listado de tareas \textit{GanttProject}.}
\label{fig:Gantt}
\end{figure}


\begin{landscape}
\begin{figure}[htpb]
\centering 
\includegraphics[height=.83\textheight]{./Figuras/Gantt-2.png}
\caption{Diagrama de Gantt según la lista de tareas.}
\label{fig:diagGantt}
\end{figure}

\end{landscape}



\section{12. Presupuesto detallado del proyecto}
\label{sec:presupuesto}

A continuación se detallan los costos asociados al proyecto, expresados en dólares estadounidenses (USD). El costo de ingeniería se estimó en función de las horas totales dedicadas, mientras que los valores de los materiales y componentes se tomaron de manera estimativa a partir de proveedores internacionales. Los costos indirectos incluyen gastos administrativos y de consumo eléctrico vinculados al desarrollo del prototipo.

\begin{table}[htpb]
\centering
\begin{tabularx}{\linewidth}{@{}|X|c|r|r|@{}}
\hline
\rowcolor[HTML]{C0C0C0} 
\multicolumn{4}{|c|}{\cellcolor[HTML]{C0C0C0}COSTOS DIRECTOS} \\ \hline
\rowcolor[HTML]{C0C0C0} 
Descripción &
  \multicolumn{1}{c|}{\cellcolor[HTML]{C0C0C0}Cantidad} &
  \multicolumn{1}{c|}{\cellcolor[HTML]{C0C0C0}Valor unitario} &
  \multicolumn{1}{c|}{\cellcolor[HTML]{C0C0C0}Valor total} \\ \hline

Trabajo directo: horas de ingeniería &
  620 &
  15 USD &
  9300 USD \\ \hline

Componentes electrónicos (MCU, sensores, etc.) &
1 &
180 USD &
180 USD \\ \hline

Placas PCB &
   1 &
   50 USD &
   50 USD \\ \hline

Materiales para la estructura &
   1 &
   60 USD &
   60 USD \\ \hline

Dispositivo de testing &
   1 &
   80 USD &
   80 USD \\ \hline

\multicolumn{3}{|c|}{SUBTOTAL} &
  9670 USD \\ \hline

\rowcolor[HTML]{C0C0C0} 
\multicolumn{4}{|c|}{\cellcolor[HTML]{C0C0C0}COSTOS INDIRECTOS} \\ \hline
\rowcolor[HTML]{C0C0C0} 
Descripción &
  \multicolumn{1}{c|}{\cellcolor[HTML]{C0C0C0}Cantidad} &
  \multicolumn{1}{c|}{\cellcolor[HTML]{C0C0C0}Valor unitario} &
  \multicolumn{1}{c|}{\cellcolor[HTML]{C0C0C0}Valor total} \\ \hline

Administración &
   50 h &
   2 USD/h &
   100 USD \\ \hline

Horas de electricidad &
   670 h &
   0.1 USD/h &
   67 USD \\ \hline

\multicolumn{3}{|c|}{SUBTOTAL} &
  167 USD \\ \hline

\rowcolor[HTML]{C0C0C0}
\multicolumn{3}{|c|}{TOTAL} &
  9837 USD \\ \hline
\end{tabularx}%
\end{table}

\section{13. Gestión de riesgos}
\label{sec:riesgos}

\textbf{a) Identificación de los riesgos y estimación de sus consecuencias:}
 
\textbf{Riesgo 1:} retrasos por disponibilidad de tiempo o carga laboral externa
\begin{itemize}
	\item Severidad: 7. Un retraso impactaría directamente en el cumplimiento del cronograma general del proyecto.
	\item Ocurrencia: 6. La dedicación disponible puede verse afectada por carga laboral concurrente.
\end{itemize}   

\textbf{Riesgo 2:} fallas en sensores o incompatibilidad con drivers
\begin{itemize}
	\item Severidad: 8. Problemas en sensores críticos (como el de peso) impedirían realizar pruebas y validar el funcionamiento del prototipo.
	\item Ocurrencia: 5. Dependiendo del fabricante, es posible encontrar variaciones en calidad, ruido eléctrico o diferencias en los drivers.
\end{itemize}

\textbf{Riesgo 3:} demoras en la adquisición de componentes electrónicos
\begin{itemize}
	\item Severidad: 6. La falta de componentes retrasaría tareas de ensamblado y pruebas iniciales del hardware.
	\item Ocurrencia: 7.El mercado local presenta disponibilidad irregular y tiempos de entrega prolongados.
\end{itemize}

\textbf{Riesgo 4:} problemas en la integración del firmware
\begin{itemize}
	\item Severidad: 9. Fallas en la integración del firmware afectarán directamente la operación global del sistema.
	\item Ocurrencia: 6. La coexistencia de múltiples periféricos (I2C, FreeRTOS, Wi-Fi/BLE) aumenta la probabilidad de errores o bloqueos.
\end{itemize}

\textbf{Riesgo 5:} fallas mecánicas o atascos en el mecanismo dispensador
\begin{itemize}
	\item Severidad: 7. Un atasco en el mecanismo impediría la correcta dispensación del alimento y comprometería la demostración final.
	\item Ocurrencia: 4. El comportamiento del alimento seco y las tolerancias mecánicas pueden generar fricción o acumulación.
\end{itemize}

\textbf{b) A continuación se puede observar la tabla de gestión de riesgos:}

\begin{table}[htpb]
\centering
\begin{tabularx}{\linewidth}{@{}|X|c|c|c|c|c|c|@{}}
\hline
\rowcolor[HTML]{C0C0C0} 
Riesgo & S & O & RPN & S* & O* & RPN* \\ \hline

Riesgo 1: retrasos por disponibilidad de tiempo & 7 & 6 & 42 & 5 & 4 & 20 \\ \hline
Riesgo 2: fallas en sensores o drivers & 8 & 5 & 40 & 6 & 4 & 24 \\ \hline
Riesgo 3: demoras en la adquisición de componentes & 6 & 7 & 42 & 4 & 5 & 20 \\ \hline
Riesgo 4: problemas en la integración del firmware & 9 & 6 & 54 & 6 & 4 & 24 \\ \hline
Riesgo 5: fallas mecánicas o atascos & 7 & 4 & 28 & 5 & 3 & 15 \\ \hline

\end{tabularx}%
\end{table}

Criterio adoptado: se tomarán medidas de mitigación en los riesgos cuyos números de RPN sean mayores o iguales a 40. A continuación se lista el plan de mitigación para cada tarea.

\textbf{c) Plan de mitigación de los riesgos que originalmente excedían el RPN máximo establecido:}
 
Riesgo 1: retrasos por disponibilidad de tiempo
  \begin{itemize}
	\item Plan de mitigación:organizar bloques fijos de trabajo semanales y mantener el avance mediante entregables parciales, priorizando tareas críticas para asegurar un progreso continuo aun con limitaciones de horario.
	\item Severidad (S*): 5. Una planificación más granular reduce el impacto global de posibles retrasos en etapas clave.
	\item Probabilidad de ocurrencia (O*): 4. Al estructurar el trabajo en sesiones breves pero constantes, disminuye la probabilidad de acumulación de tareas y desfasajes significativos.
	\end{itemize}

Riesgo 2: fallas en sensores o drivers
  \begin{itemize}
	\item Plan de mitigación: realizar pruebas unitarias tempranas sobre cada sensor y módulo de driver, mantener una biblioteca de ejemplos de referencia y disponer de repuestos de los componentes más críticos.
	\item Severidad (S*): 6. Las pruebas aisladas permiten detectar fallas antes de la integración, reduciendo su impacto en el desarrollo del sistema.
	\item Probabilidad de ocurrencia (O*): 4. Contar con repuestos y validaciones previas disminuye la probabilidad de interrupciones prolongadas durante la implementación.
	\end{itemize}


Riesgo 4: problemas en la integración del firmware y validación del sistema
  \begin{itemize}
	\item Plan de mitigación: establecer hitos semanales de avance y realizar pruebas incrementales sobre cada módulo del firmware para detectar fallas tempranas.
	\item Severidad (S*): 6. Una integración progresiva reduce el impacto global ante errores funcionales.
	\item Probabilidad de ocurrencia (O*): 4. Una integración incremental con pruebas intermedias reduce la posibilidad de errores acumulados entre módulos, disminuyendo la probabilidad de fallas complejas en etapas avanzadas.
	\end{itemize}


\section{14. Gestión de la calidad}
\label{sec:calidad}

A continuación se listan los principales requerimientos del proyecto, junto con sus acciones de verificación y validación para cada uno:

\begin{itemize} 
\item Req \#1.1: el dispositivo deberá dispensar alimento de manera controlada y consistente.

\begin{itemize}
	\item Se verificará que el motor y el mecanismo de apertura cumplan con los valores de torque, velocidad y consumo especificados para un dispensado estable.
	\item Se validará que la cantidad de alimento liberada sea uniforme mediante pruebas repetitivas bajo distintas condiciones de operación.
\end{itemize}

\end{itemize}

\begin{itemize} 
\item Req \#1.2: el sistema deberá detectar el nivel de alimento mediante sensores adecuados.

\begin{itemize}
	\item Se verificará la documentación técnica del sensor elegido, corroborando su precisión y rango de medición.
	\item Se validará la lectura del nivel comparando las mediciones con valores reales en múltiples ciclos, registrando su variación y estabilidad.
\end{itemize}

\end{itemize}

\begin{itemize} 
\item Req \#1.3: el dispositivo deberá conectarse correctamente a la red Wi-Fi para permitir control y monitoreo remoto.

\begin{itemize}
	\item Se verificará que el módulo Wi-Fi cumpla con el estándar 802.11 b/g/n y que su integración eléctrica y de firmware sea correcta.
	\item Se validará que el dispositivo pueda enviar y recibir datos de manera sostenida, manteniendo una conexión estable con la aplicación o servidor.
\end{itemize}

\end{itemize}

\vspace{2.5cm}

\begin{itemize} 
\item Req \#1.4: el sistema deberá almacenar localmente datos del funcionamiento.

\begin{itemize}
	\item Se verificará que la memoria Flash cumpla con los requisitos de capacidad, integridad y tiempos de acceso.
	\item Se validará que la información almacenada se registre y recupere correctamente durante la ejecución, sin pérdidas de datos.
\end{itemize}

\end{itemize}

\begin{itemize} 
\item Req \#1.5: el firmware deberá operar bajo FreeRTOS con correcta gestión de tareas y sincronización.

\begin{itemize}
	\item Se verificará la asignación de prioridades, uso de colas, semáforos y temporizadores según la arquitectura definida.
	\item Se validará la estabilidad del sistema ejecutando pruebas con múltiples tareas concurrentes, interrupciones frecuentes y escenarios de carga elevada.
\end{itemize}

\end{itemize}

\begin{itemize} 
\item Req \#1.6: el dispositivo deberá operar de manera segura para el usuario y para los animales.

\begin{itemize}
	\item Se verificará que los componentes eléctricos cumplan con normas de baja tensión (5–12 V) y que las partes móviles se encuentren adecuadamente cubiertas.
	\item Se validará que el equipo pueda manipularse sin riesgo, comprobando que no haya bordes filosos ni mecanismos que puedan generar atrapamientos.
\end{itemize}

\end{itemize}

\begin{itemize} 
\item Req \#1.7: el sistema deberá ser capaz de recuperarse tras un corte breve de energía.

\begin{itemize}
	\item Se verificará la implementación de mecanismos de backup, tales como RTC, memoria persistente o restauración de estados.
	\item Se validará que, ante una interrupción simulada de alimentación, el dispositivo retome su funcionamiento sin corrupción de datos ni pérdida del estado previo.
\end{itemize}

\end{itemize}

\section{15. Procesos de cierre}    
\label{sec:cierre}

A continuación se establecen las pautas para realizar la evaluación final del proyecto, contemplando las siguientes actividades:

\begin{enumerate}
\item Pautas de trabajo que se seguirán para analizar si se respetó el Plan de Proyecto original:
	\begin{itemize}
		\item Responsable: \authorname .
		\item Se realizará un análisis comparativo entre las tareas planificadas y las realmente ejecutadas, identificando desvíos en tiempos, orden y carga de trabajo.
		\item Se evaluará si los objetivos del proyecto fueron cumplidos conforme al plan original.
		\item Se documentarán las causas principales de los desvíos y se propondrán medidas de mejora para futuros proyectos.
	\end{itemize}
	
\vspace{2.5cm}

\item Identificación de las técnicas y procedimientos útiles e inútiles que se emplearon, y los problemas que surgieron y cómo se solucionaron:
	\begin{itemize}
		\item Responsable: \authorname .
		\item Se realizará un análisis de las técnicas, herramientas y procedimientos utilizados, identificando cuáles resultaron útiles, prescindibles o ineficientes durante el desarrollo.
		\item Se documentarán los problemas que surgieron en la implementación y las soluciones adoptadas, junto con una valoración del impacto real de cada herramienta empleada.
		\item Se identificarán oportunidades de mejora y la posible incorporación de nuevas técnicas o procedimientos para proyectos futuros.
	\end{itemize}
	
\item Organización de presentación del proyecto:
	\begin{itemize}
		\item Responsable: \authorname .
		\item Se invitará a la presentación del proyecto final a todas las personas vinculadas o colaboradoras en el proyecto, y se agradecerá al director del proyecto, a jurados, docentes, compañeros y autoridades de la Carrera de Especialización en Sistemas Embebidos.
	\end{itemize}
	
\end{enumerate}

\end{document}