\documentclass[
11pt, % The default document font size, options: 10pt, 11pt, 12pt
%codirector, % Uncomment to add a codirector to the title page
]{charter} 


% El títulos de la memoria, se usa en la carátula y se puede usar el cualquier lugar del documento con el comando \ttitle
\titulo{Dispensador inteligente de comida para gatos} 

% Nombre del posgrado, se usa en la carátula y se puede usar el cualquier lugar del documento con el comando \degreename
\posgrado{Carrera de Especialización en Sistemas Embebidos} 

% Tu nombre, se puede usar el cualquier lugar del documento con el comando \authorname
% IMPORTANTE: no omitir titulaciones ni tildación en los nombres, también se recomienda escribir los nombres completos (tal cual los tienen en su documento)
\autor{Ing. Durante Matías Nahuel}

% El nombre del director y co-director, se puede usar el cualquier lugar del documento con el comando \supname y \cosupname y \pertesupname y \pertecosupname
\director{Ing. Bualó Santiago}
\pertenenciaDirector{FIUBA} 
\codirector{} % para que aparezca en la portada se debe descomentar la opción codirector en los parámetros de documentclass
\pertenenciaCoDirector{FIUBA}

% Nombre del cliente, quien va a aprobar los resultados del proyecto, se puede usar con el comando \clientename y \empclientename
\cliente{Nombre del cliente}
\empresaCliente{Empresa del cliente}
 
\fechaINICIO{21 de octubre de 2025}		%Fecha de inicio de la cursada de GdP \fechaInicioName
\fechaFINALPlan{14 de diciembre de 2025} 	%Fecha de final de cursada de GdP
\fechaFINALTrabajo{20 de julio de 2026}	%Fecha de defensa pública del trabajo final

\newcommand{\horaTrabajo}{620}

\begin{document}

\maketitle
\thispagestyle{empty}
\pagebreak


\thispagestyle{empty}
{\setlength{\parskip}{0pt}
\tableofcontents{}
}
\pagebreak


\section*{Registros de cambios}
\label{sec:registro}


\begin{table}[ht]
\label{tab:registro}
\centering
\begin{tabularx}{\linewidth}{@{}|c|X|c|@{}}
\hline
\rowcolor[HTML]{C0C0C0} 
Revisión & \multicolumn{1}{c|}{\cellcolor[HTML]{C0C0C0}Detalles de los cambios realizados} & Fecha      \\ \hline
0      & Creación del documento                                 &\fechaInicioName \\ \hline
1      & Primera entrega               							& {04} de {noviembre} de 2025 \\ \hline
2      & Segunda entrega                                          & {11} de {noviembre} de 2025 \\ \hline
%3      & Se completa hasta el punto 12 inclusive                & {día} de {mes} de 202X \\ \hline
%4      & Se completa el plan	                                 & {día} de {mes} de 202X \\ \hline

% Si hay más correcciones pasada la versión 4 también se deben especificar acá

\end{tabularx}
\end{table}

\pagebreak



\section*{Acta de constitución del proyecto}
\label{sec:acta}

\begin{flushright}
Buenos Aires, \fechaInicioName
\end{flushright}

\vspace{2cm}

Por medio de la presente se acuerda con el \authorname\hspace{1px} que su Trabajo Final de la \degreename\hspace{1px} se titulará ``\ttitle'' y consistirá en la implementación de un prototipo de un sistema de dispensador inteligante de comida para gatos, capaz de automatizar la entrega de alimento seco en horarios programados, monitorear el consumo y reportar la información a una plataforma web, y tendrá un presupuesto preliminar estimado de {\horaTrabajo} horas y un costo estimado de \textcolor{red}{\$ 8000}, con fecha de inicio el \fechaInicioName\hspace{1px} y fecha de presentación pública el \fechaFinalName.

Se adjunta a esta acta la planificación inicial.

\vfill

\begin{table}[ht]
\centering
\begin{tabular}{ccc}
\begin{tabular}[c]{@{}c@{}}Dr. Ing. Ariel Lutenberg \\ Director posgrado FIUBA\end{tabular} &  \vspace{2.5cm} \\ 
\multicolumn{3}{c}{\begin{tabular}[c]{@{}c@{}} \supname \\ Director del Trabajo Final\end{tabular}} \vspace{2.5cm} \\
\end{tabular}
\end{table}




\section{1. Descripción técnica-conceptual del proyecto a realizar}
\label{sec:descripcion}

El presente trabajo práctico busca implementar un sistema de alimentación inteligente para mascotas, capaz de administrar y controlar de manera automática la entrega de alimento, supervisando su consumo y registrando la información correspondiente. El sistema tiene como finalidad facilitar el cuidado de los gatos en el hogar, asegurando que reciban la cantidad adecuada de comida en los horarios establecidos, incluso en ausencia de los dueños.

Se ha observado que una de las problemáticas más comunes entre los propietarios de mascotas es la dificultad para garantizar una alimentación constante y controlada, especialmente cuando no se encuentran en el hogar debido a largas jornadas laborales o deben ausentarse por períodos prolongados. En muchos casos, se depende de terceros o de dispensadores automáticos simples que no brindan información sobre si el animal efectivamente se alimentó, ni permiten llevar un registro del consumo diario. El objetivo principal de este proyecto es dar una solución tecnológica a esta necesidad, aplicando principios de automatización y monitoreo inteligente, a fin de mejorar la calidad de vida tanto de las mascotas como de sus dueños.

El sistema a desarrollar se basará en un conjunto de sensores que permitirán controlar la cantidad de alimento dispensado y verificar su consumo, junto con un mecanismo automatizado que permitirá dosificar las raciones según los horarios programados. Además, el sistema incluirá un registro continuo de datos que posibilitará identificar patrones y comportamientos alimentarios de la mascota, aportando información valiosa para la evaluación y mejora del plan de alimentación junto a profesionales del área.
El dispositivo tendrá la capacidad de funcionar de manera autónoma, controlando la entrega del alimento, detectando si la ración fue consumida, y comunicándose con una interfaz externa, desde la cual el usuario podrá visualizar reportes, modificar horarios o recibir alertas cuando el depósito de alimento esté próximo a vaciarse.

Actualmente, existen en el mercado diversos dispositivos que permiten automatizar la entrega de comida para mascotas, pero la mayoría presenta limitaciones en cuanto al monitoreo del consumo o la disponibilidad de información remota. En la mayoría de los casos, el usuario no puede saber si el animal efectivamente comió o si se produjo alguna falla en la dispensación.
La propuesta de este proyecto busca superar esas limitaciones, ofreciendo una solución que combina la automatización, el sensado, el control inteligente y la conectividad. De esta forma, no solo se garantiza la alimentación adecuada del animal, sino que además se obtiene información confiable y útil para su seguimiento.

La iniciativa también pretende sentar las bases de un nuevo enfoque en el cuidado automatizado de mascotas, incorporando tecnologías que hasta el momento se aplicaban principalmente en entornos industriales o de domótica. Así, se impulsa un paradigma en el que la automatización del hogar se extiende al bienestar animal, contribuyendo a una gestión más eficiente del tiempo y a una mejor calidad de vida en los hogares.
Con esta nueva gestión automatizada de la alimentación, se podrán mejorar los siguientes aspectos:

\begin{itemize}
\item Mantener un control preciso de las raciones diarias y del consumo efectivo de la mascota.
\item Establecer rutinas automáticas y personalizadas de alimentación.
\item Generar reportes de consumo útiles para el seguimiento nutricional.
\end{itemize}

\vspace{2.5cm}

Para llevar a cabo este proyecto, se propone un módulo integral que gestione de forma automática el almacenamiento, dispensado y monitoreo del alimento. Dicho módulo será capaz de administrar el proceso completo de alimentación, registrando la información de consumo y enviándola a una interfaz de usuario desde la cual el propietario podrá consultar el estado del sistema, modificar configuraciones o recibir alertas en caso de que el alimento disponible se encuentre próximo a agotarse.
El sistema será autónomo, de fácil instalación y adaptable a diferentes entornos domésticos, de modo que pueda operar sin depender de una ubicación fija ni de supervisión constante.

A continuación, se presenta en la Figura \ref{fig:diagBloques} un diagrama en bloques del sistema propuesto, en el cual se observan los principales módulos que lo componen:

\begin{figure}[htpb]
\centering 
\includegraphics[width=.65\textwidth]{./Figuras/diagBloques.pdf}
\caption{Diagrama en bloques del sistema.}
\label{fig:diagBloques}
\end{figure}


\section{2. Identificación y análisis de los interesados}
\label{sec:interesados}
A continuación se listan todas las partes involucradas en el proyecto:

\begin{table}[ht]
\begin{tabularx}{\linewidth}{@{}|l|X|X|l|@{}}
\hline
\rowcolor[HTML]{C0C0C0} 
Rol           & Nombre y Apellido & Organización 	& Puesto 	\\ \hline
Responsable   & \authorname       & FIUBA        	& Alumno 	\\ \hline
Orientador    & \supname	      & \pertesupname 	& Director del Trabajo Final \\ \hline
Usuario final & Dueños de mascotas &              	&        	\\ \hline
\end{tabularx}
\caption{Identificación de los interesados} 
\label{tab:interesados}
\end{table}


\vspace{2.5cm}

\begin{itemize}
	\item \textbf{Responsable:} Ing. Durante Matías Nahuel, quien será el responsable de la planificación, diseño, desarrollo e implementación del sistema propuesto, asumiendo además todos los gastos y/o beneficios económicos al tratarse de un proyecto de carácter personal.
	\item \textbf{Orientador:}  Ing. Bualó Santiago, quien actuará como director del trabajo final, brindando acompañamiento y supervisión técnica durante el desarrollo del proyecto.
	\item \textbf{Usuario final:} dado que el proyecto se orienta a ofrecer una solución para el cuidado automatizado de mascotas, los usuarios finales serán todas aquellas personas que posean gatos y deseen optimizar su alimentación diaria mediante un sistema automático y monitoreado.
\end{itemize}

\section{3. Propósito del proyecto}
\label{sec:proposito}

El propósito de este proyecto es aplicar tecnologías de automatización y monitoreo inteligente al cuidado doméstico de mascotas, mediante el desarrollo de un sistema confiable y adaptable que administre, registre y analice la alimentación de gatos, optimizando la interacción entre el usuario y su mascota.

\section{4. Alcance del proyecto}
\label{sec:alcance}

Dentro del alcance del proyecto se contemplan las siguientes tareas principales:
\begin{itemize}
	\item El desarrollo de un dispositivo capaz de automatizar la alimentación de gatos, permitiendo dispensar el alimento mediante los siguientes modos:
		\begin{itemize}
		\item \textbf{Automático:} mediante la programación de horarios y raciones predefinidas.
		\item \textbf{Manual}: a través de un pulsador que permitirá activar el dispensado del alimento.
		\item \textbf{Remoto (opcional)}: mediante una aplicación móvil o conexión web, que posibilitará la gestión y el monitoreo del sistema a distancia.
		\end{itemize}
	\item El diseño del sistema de sensado, encargado de controlar el nivel de alimento disponible tanto en el depósito como en el plato.
	\item La implementación de un sistema de registro de datos que permita almacenar información sobre las raciones dispensadas y el consumo de alimento.
	\item El desarrollo de una interfaz externa al dispositivo, desde la cual el usuario podrá visualizar el estado del sistema, modificar los horarios de alimentación y recibir alertas ante fallas o niveles bajos de alimento.
	\item El sistema de alimentación propio del prototipo.
	
\end{itemize}

No serán parte del proyecto los siguientes:

\begin{itemize}
	\item El desarrollo de una aplicación móvil o plataforma web completa.
	\item La incorporación de identificación individual de las mascotas.
	\item La utilización de servicios de almacenamiento externo o en la nube.
	\item El desarrollo de un prototipo final destinado a su comercialización.
	\item El diseño de la estructura definitiva del dispositivo.
\end{itemize}

\section{5. Supuestos del proyecto}
\label{sec:supuestos}

Para el desarrollo del presente proyecto se supone que:

\begin{itemize}
	\item Se dispondrá de los recursos económicos para la adquisición de materiales y componentes requeridos.
	\item Los componentes electrónicos y mecánicos necesarios estarán disponibles en el mercado local.
	\item El dispositivo podrá ser probado en un entorno doméstico con suministro eléctrico.
	\item No habrá restricciones reglamentarias, técnicas o logísticas que impidan el desarrollo y la validación del prototipo.
	\item El alimento utilizado será de tipo seco, con formato y tamaño compatibles con el mecanismo de dosificación a implementar.
	
\end{itemize}

\section{6. Requerimientos}
\label{sec:requerimientos}
A continuación se enumeran los requerimientos del proyecto:
\begin{enumerate}
	\item Requerimientos funcionales:
		\begin{enumerate}
			\item El sistema debe dispensar alimento seco de manera automática en horarios programados.
			\item El sistema debe permitir el dispensado manual mediante un botón físico.
			\item El sistema debe registrar y almacenar los datos de consumo diario.
			\item El sistema debe detectar el nivel de alimento disponible en el tanque.
			\item El sistema debe medir el peso del alimento dispensado en el plato.
			\item El sistema debe emitir una alerta cuando el nivel de alimento del tanque sea bajo.
			\item El sistema debe ser capaz de operar de forma autónoma sin conexión a internet.
			\item El sistema debe permitir la visualización de reportes e historial de consumo desde una interfaz web o aplicación.
		\end{enumerate}
	\item Requerimientos de hardware:
		\begin{enumerate}
			\item El sistema deberá incorporar una placa microcontroladora de la familia STM, a fin de mantener la compatibilidad con las bibliotecas HAL (Hardware Abstraction Layer) utilizadas en el desarrollo del firmware.
			\item El sensado de peso se realizará mediante módulos HX711 y celdas de carga.
			\item El sistema deberá incorporar un módulo Wi-Fi (ESP8266 o ESP32) para la comunicación remota.
			\item El sistema deberá contar con un botón físico para el accionamiento manual.
			\item El prototipo deberá alimentarse mediante una fuente externa.
		\end{enumerate}
	\item Requerimiento de firmware:
		\begin{enumerate}
			\item El firmware deberá ser desarrollado en lenguaje C.
			\item Se deberá integrar un sistema operativo en tiempo real (FreeRTOS) para la gestión de tareas concurrentes.
			\item El sistema deberá incluir rutinas de verificación de sensores y comunicación, realizando pruebas unitarias e integrales para cada módulo.
		\end{enumerate}
	\item Requerimientos de gestión:
		\begin{enumerate}
			\item Se utilizará un repositorio público (GitHub) para el control de versiones y seguimiento del código.
			\item El tiempo total estimado para el desarrollo del prototipo será de {\horaTrabajo} horas.
			\item Se establecerán reuniones periódicas de seguimiento con el director del trabajo final para evaluar el progreso del proyecto.
		\end{enumerate}
\end{enumerate}





\section{7. Historias de usuarios (\textit{Product backlog})}
\label{sec:backlog}

En esta sección se enuncian las historias de usuario, asignándoles un puntaje según los siguientes aspectos:
\begin{enumerate}
	\item Dificultad del trabajo.
	\item Complejidad del trabajo.
	\item Riesgo asociado.
\end{enumerate}

Para asignar la ponderación, se utiliza una escala basada en la serie de Fibonacci, donde un número mayor implica un mayor costo.
El Story Point se calcula sumando las tres ponderaciones, y en caso de que el resultado no pertenezca a la serie, se redondea al número superior más cercano.

\begin{enumerate}
	\item Como dueño de mascota quiero poder programar los horarios de comida para asegurar que mi gato reciba su alimento de forma regular y sin depender de mi presencia.
		\begin{itemize}
			\item Dificultad: 5
			\item Complejidad: 5
			\item Riesgo: 8
			\item Story Point: 21
		\end{itemize}
	\item Como usuario quiero recibir una alerta cuando el tanque de alimento esté por terminarse, para poder recargarlo antes de que se quede sin comida.
		\begin{itemize}
			\item Dificultad: 8
			\item Complejidad: 5
			\item Riesgo: 3
			\item Story Point: 21
		\end{itemize}
	\item Como usuario quiero disponer en cualquier momento de la información sobre si mi gato consumió su comida, a través de la medición del peso del plato.	
		\begin{itemize}
			\item Dificultad: 8
			\item Complejidad: 8
			\item Riesgo: 5
			\item Story Point: 34
		\end{itemize}
	\item Como usuario quiero poder dispensar alimento manualmente, para darle comida a mi gato fuera de los horarios programados.
		\begin{itemize}
			\item Dificultad: 3
			\item Complejidad: 3
			\item Riesgo: 2
			\item Story Point: 8
		\end{itemize}
	\item Como usuario quiero consultar reportes sobre el consumo diario y semanal, para conocer los hábitos alimenticios de mi gato.
		\begin{itemize}
			\item Dificultad: 8
			\item Complejidad: 13
			\item Riesgo: 8
			\item Story Point: 34
		\end{itemize}	
	\item Como usuario quiero que el sistema funcione incluso sin conexión a internet, garantizando el dispensado automático aunque haya fallas en la red Wi-Fi.
		\begin{itemize}
			\item Dificultad: 5
			\item Complejidad: 8
			\item Riesgo: 8
			\item Story Point: 21
		\end{itemize}				
\end{enumerate}



\section{8. Entregables principales del proyecto}
\label{sec:entregables}

Los entregables del proyecto son:

\begin{itemize}
	\item Diagrama de bloques del sistema propuesto.
	\item Diseño esquemático del circuito electrónico.
	\item Simulación o integración del hardware en entorno de prueba.
	\item Código fuente del firmware desarrollado en lenguaje C.
	\item Repositorio con documentación técnica y control de versiones.
	\item Documentación técnica y hojas de datos de las placas y módulos utilizados.
	\item Guía de usuario.
\end{itemize}

\section{9. Desglose del trabajo en tareas}
\label{sec:wbs}

A continuación se detallan las tareas y sub-tareas, junto con su duración estimada en horas de trabajo.

\begin{enumerate}
\item Planificación y gestión del proyecto (80 h)
	\begin{enumerate}
	\item Elaboración del plan de trabajo (10 h)
	\item Definición de componentes y proveedores (15 h)
	\item Planificación del desarrollo de firmware y hardware (20 h)
	\item Redacción de informes y presentaciones (25 h)
	\item Elaboración de guía de usuario y otros documentos afines (10 h)
	\end{enumerate}
\item Investigación y análisis técnico (130 h)
	\begin{enumerate}
	\item Relevamiento de dispensadores automáticos existentes en el mercado(20 h)
	\item Selección y análisis de sensores de peso (25 h)
	\item Evaluación de opciones de conectividad (20 h)
	\item Evaluación de métodos de dispensado de alimento (15 h)
	\item Estudio de librerías de FreeRTOS y drivers compatibles (20 h)
	\item Análisis de estructura mecánica del dispensador (10 h)
	\item Investigación de protocolos Wi-Fi y BLE para comunicación del sistema (20 h)
	\end{enumerate}
\item Desarrollo de hardware (160 h)
	\begin{enumerate}
	\item Diseño esquemático del circuito (30 h)
	\item Diseño del PCB y generación de archivos de fabricación (40 h)
	\item Ensamblado y soldadura de componentes (25 h)
	\item Diseño y construcción de la estructura mecánica del dispensador (35 h)
	\item Pruebas eléctricas y de funcionamiento inicial (30 h)
	\end{enumerate}
\item Desarrollo de Software (180 h)	
	\begin{enumerate}
	\item  Desarrollo de drivers específicos para cada uno de los periféricos (35 h)
	\item Implementación de la máquina de estados y funcionamiento del dispensado (30 h)
	\item Integración de FreeRTOS y gestión de tareas (40 h)
	\item Implementación de comunicación Wi-Fi con servidor (30 h)
	\item Desarrollo de registro de datos y almacenamiento local (20 h)
	\item Optimización del código y pruebas de funcionamiento (25 h)	
	\end{enumerate}	
\item Fabricación y armado del prototipo (70 h)	
	\begin{enumerate}
	\item Fabricación o montaje del PCB (10 h)
	\item Adquisición e integración de componentes (10 h)
	\item Ensamblado general del dispositivo (20 h)
	\item Pruebas funcionales y ajustes mecánicos (30 h)
	\end{enumerate}
\end{enumerate}

Cantidad total de horas: {\horaTrabajo} h

\section{10. Diagrama de Activity On Node}
\label{sec:AoN}

\begin{consigna}{red}
Armar el AoN a partir del WBS definido en la etapa anterior.

Una herramienta simple para desarrollar los diagramas es el Draw.io (\url{https://app.diagrams.net/}).
\href{https://app.diagrams.net}{Draw.io}


\begin{figure}[htpb]
\centering 
\includegraphics[width=.8\textwidth]{./Figuras/AoN.png}
\caption{Diagrama de \textit{Activity on Node}.}
\label{fig:AoN}
\end{figure}

Indicar claramente en qué unidades están expresados los tiempos.
De ser necesario indicar los caminos semi críticos y analizar sus tiempos mediante un cuadro.
Es recomendable usar colores y un cuadro indicativo describiendo qué representa cada color.

\end{consigna}

\section{11. Diagrama de Gantt}
\label{sec:gantt}

\begin{consigna}{red}
Existen muchos programas y recursos \textit{online} para hacer diagramas de Gantt, entre los cuales destacamos:

\begin{itemize}
\item Planner
\item GanttProject
\item Trello + \textit{plugins}. En el siguiente link hay un tutorial oficial: \\ \url{https://blog.trello.com/es/diagrama-de-gantt-de-un-proyecto}
\item Creately, herramienta online colaborativa. \\\url{https://creately.com/diagram/example/ieb3p3ml/LaTeX}
\item Se puede hacer en latex con el paquete \textit{pgfgantt}\\ \url{http://ctan.dcc.uchile.cl/graphics/pgf/contrib/pgfgantt/pgfgantt.pdf}
\end{itemize}

Pegar acá una captura de pantalla del diagrama de Gantt, cuidando que la letra sea suficientemente grande como para ser legible. 
Si el diagrama queda demasiado ancho, se puede pegar primero la ``tabla'' del Gantt y luego pegar la parte del diagrama de barras del diagrama de Gantt.

Configurar el software para que en la parte de la tabla muestre los códigos del EDT (WBS).\\
Configurar el software para que al lado de cada barra muestre el nombre de cada tarea.\\
Revisar que la fecha de finalización coincida con lo indicado en el Acta Constitutiva.

En la figura \ref{fig:gantt}, se muestra un ejemplo de diagrama de gantt realizado con el paquete de \textit{pgfgantt}. 
En la plantilla pueden ver el código que lo genera y usarlo de base para construir el propio.

Las fechas pueden ser calculadas utilizando alguna de las herramientas antes citadas. Sin embargo, el siguiente ejemplo
fue elaborado utilizando 
\href{https://docs.google.com/spreadsheets/d/1fBz8NhSpc4tkkhz3KjJCbh1nR_ltDkfEcZi4tZXduqs}{esta hoja de cálculo}.

Es importante destacar que el ancho del diagrama estará dado por la longitud del texto utilizado para las tareas 
(Ejemplo: tarea 1, tarea 2, etcétera) y el valor \textit{x unit}. Para mejorar la apariencia del diagrama, es necesario
ajustar este valor y, quizás, acortar los nombres de las tareas.

\begin{figure}[htpb]
  \begin{center}
    \begin{ganttchart}[
      time slot unit=day,
      time slot format=isodate,
      x unit=0.038cm,
      y unit title=0.7cm,
      y unit chart=0.6cm,
      milestone/.append style={xscale=4}
      ]{2021-03-05}{2021-12-16}
      \gantttitlecalendar*{2021-03-05}{2021-12-16}{year} \\
      \gantttitlecalendar*{2021-03-05}{2021-12-16}{month} \\
      \ganttgroup{Duración Total}{2021-03-05}{2021-12-16} \\
      %%%%%%%%%%%%%%%%%Organización
      \ganttgroup{Organización}{2021-03-05}{2021-04-16} \\
      \ganttbar{Planificación del proyecto}{2021-03-05}{2021-04-15} \\
      %%%%%%%%%%%%%%%%%Ejecución
      \ganttgroup{Ejecución}{2021-04-16}{2021-10-21} \\
      \ganttbar{Tarea 1}{2021-04-16}{2021-04-29} \\
      \ganttbar{Tarea 2}{2021-04-30}{2021-05-13} \\
      \ganttbar{Tarea 3}{2021-05-14}{2021-05-27} \\
      \ganttbar{Tarea 4}{2021-05-28}{2021-07-12} \\
      \ganttbar{Tarea 5}{2021-07-13}{2021-08-09} \\
      \ganttbar{Tarea 6}{2021-08-10}{2021-09-23} \\
      \ganttbar{Tarea 7}{2021-09-24}{2021-09-30} \\
      \ganttbar{Tarea 8}{2021-10-01}{2021-10-14} \\
      \ganttbar{Tarea 9}{2021-10-15}{2021-10-21} \\
      % %%%%%%%%%%%%%%%%%Finalización
      \ganttgroup{Finalización}{2021-10-22}{2021-12-16} \\
      \ganttbar{Memoria v1}{2021-10-22}{2021-11-04} \\
      \ganttbar{Memoria v2}{2021-11-05}{2021-11-18} \\
      \ganttbar{Memoria final}{2021-11-19}{2021-12-02} \\
      % La fecha del siguiente milestone es la fecha en que terminamos la memoria
      \ganttmilestone{Enviar memoria al director}{2021-12-02} \\
      \ganttbar{Elaborar la presentación}{2021-12-03}{2021-12-16} \\
      \ganttmilestone{Ensayo de la presentación}{2021-12-16} \\
      %%%%%%%%%%%%%%%%%%%%%%%%%%%%%%%%%%%%%%%%%%%%%%%%%%%%%%%%%%%%%%%
    \end{ganttchart}
  \end{center}
  \caption{Diagrama de gantt de ejemplo}
  \label{fig:gantt}
\end{figure}


\begin{landscape}
\begin{figure}[htpb]
\centering 
\includegraphics[height=.85\textheight]{./Figuras/Gantt-2.png}
\caption{Ejemplo de diagrama de Gantt (apaisado).} %Modificar este título acorde.
\label{fig:diagGantt}
\end{figure}

\end{landscape}

\end{consigna}


\section{12. Presupuesto detallado del proyecto}
\label{sec:presupuesto}

\begin{consigna}{red}
Si el proyecto es complejo entonces separarlo en partes:
\begin{itemize}
	\item Un total global, indicando el subtotal acumulado por cada una de las áreas.
	\item El desglose detallado del subtotal de cada una de las áreas.
\end{itemize}

IMPORTANTE: No olvidarse de considerar los COSTOS INDIRECTOS.

Incluir la aclaración de si se emplea como moneda el peso argentino (ARS) o si se usa moneda extranjera (USD, EUR, etc). Si es en moneda extranjera se debe indicar la tasa de conversión respecto a la moneda local en una fecha dada.

\end{consigna}

\begin{table}[htpb]
\centering
\begin{tabularx}{\linewidth}{@{}|X|c|r|r|@{}}
\hline
\rowcolor[HTML]{C0C0C0} 
\multicolumn{4}{|c|}{\cellcolor[HTML]{C0C0C0}COSTOS DIRECTOS} \\ \hline
\rowcolor[HTML]{C0C0C0} 
Descripción &
  \multicolumn{1}{c|}{\cellcolor[HTML]{C0C0C0}Cantidad} &
  \multicolumn{1}{c|}{\cellcolor[HTML]{C0C0C0}Valor unitario} &
  \multicolumn{1}{c|}{\cellcolor[HTML]{C0C0C0}Valor total} \\ \hline
 &
  \multicolumn{1}{c|}{} &
  \multicolumn{1}{c|}{} &
  \multicolumn{1}{c|}{} \\ \hline
 &
  \multicolumn{1}{c|}{} &
  \multicolumn{1}{c|}{} &
  \multicolumn{1}{c|}{} \\ \hline
\multicolumn{1}{|l|}{} &
   &
   &
   \\ \hline
\multicolumn{1}{|l|}{} &
   &
   &
   \\ \hline
\multicolumn{3}{|c|}{SUBTOTAL} &
  \multicolumn{1}{c|}{} \\ \hline
\rowcolor[HTML]{C0C0C0} 
\multicolumn{4}{|c|}{\cellcolor[HTML]{C0C0C0}COSTOS INDIRECTOS} \\ \hline
\rowcolor[HTML]{C0C0C0} 
Descripción &
  \multicolumn{1}{c|}{\cellcolor[HTML]{C0C0C0}Cantidad} &
  \multicolumn{1}{c|}{\cellcolor[HTML]{C0C0C0}Valor unitario} &
  \multicolumn{1}{c|}{\cellcolor[HTML]{C0C0C0}Valor total} \\ \hline
\multicolumn{1}{|l|}{} &
   &
   &
   \\ \hline
\multicolumn{1}{|l|}{} &
   &
   &
   \\ \hline
\multicolumn{1}{|l|}{} &
   &
   &
   \\ \hline
\multicolumn{3}{|c|}{SUBTOTAL} &
  \multicolumn{1}{c|}{} \\ \hline
\rowcolor[HTML]{C0C0C0}
\multicolumn{3}{|c|}{TOTAL} &
   \\ \hline
\end{tabularx}%
\end{table}


\section{13. Gestión de riesgos}
\label{sec:riesgos}

\begin{consigna}{red}
a) Identificación de los riesgos (al menos cinco) y estimación de sus consecuencias:
 
Riesgo 1: detallar el riesgo (riesgo es algo que si ocurre altera los planes previstos de forma negativa)
\begin{itemize}
	\item Severidad (S): mientras más severo, más alto es el número (usar números del 1 al 10).\\
	Justificar el motivo por el cual se asigna determinado número de severidad (S).
	\item Probabilidad de ocurrencia (O): mientras más probable, más alto es el número (usar del 1 al 10).\\
	Justificar el motivo por el cual se asigna determinado número de (O). 
\end{itemize}   

Riesgo 2:
\begin{itemize}
	\item Severidad (S): X.\\
	Justificación...
	\item Ocurrencia (O): Y.\\
	Justificación...
\end{itemize}

Riesgo 3:
\begin{itemize}
	\item Severidad (S):  X.\\
	Justificación...
	\item Ocurrencia (O): Y.\\
	Justificación...
\end{itemize}


b) Tabla de gestión de riesgos:      (El RPN se calcula como RPN=SxO)

\begin{table}[htpb]
\centering
\begin{tabularx}{\linewidth}{@{}|X|c|c|c|c|c|c|@{}}
\hline
\rowcolor[HTML]{C0C0C0} 
Riesgo & S & O & RPN & S* & O* & RPN* \\ \hline
       &   &   &     &    &    &      \\ \hline
       &   &   &     &    &    &      \\ \hline
       &   &   &     &    &    &      \\ \hline
       &   &   &     &    &    &      \\ \hline
       &   &   &     &    &    &      \\ \hline
\end{tabularx}%
\end{table}

Criterio adoptado: 

Se tomarán medidas de mitigación en los riesgos cuyos números de RPN sean mayores a...

Nota: los valores marcados con (*) en la tabla corresponden luego de haber aplicado la mitigación.

c) Plan de mitigación de los riesgos que originalmente excedían el RPN máximo establecido:
 
Riesgo 1: plan de mitigación (si por el RPN fuera necesario elaborar un plan de mitigación).
  Nueva asignación de S y O, con su respectiva justificación:
  \begin{itemize}
	\item Severidad (S*): mientras más severo, más alto es el número (usar números del 1 al 10).
          Justificar el motivo por el cual se asigna determinado número de severidad (S).
	\item Probabilidad de ocurrencia (O*): mientras más probable, más alto es el número (usar del 1 al 10).
          Justificar el motivo por el cual se asigna determinado número de (O).
	\end{itemize}

Riesgo 2: plan de mitigación (si por el RPN fuera necesario elaborar un plan de mitigación).
 
Riesgo 3: plan de mitigación (si por el RPN fuera necesario elaborar un plan de mitigación).

\end{consigna}


\section{14. Gestión de la calidad}
\label{sec:calidad}

\begin{consigna}{red}
Elija al menos diez requerimientos que a su criterio sean los más importantes/críticos/que aportan más valor y para cada uno de ellos indique las acciones de verificación y validación que permitan asegurar su cumplimiento.

\begin{itemize} 
\item Req \#1: copiar acá el requerimiento con su correspondiente número.

\begin{itemize}
	\item Verificación para confirmar si se cumplió con lo requerido antes de mostrar el sistema al cliente. Detallar.
	\item Validación con el cliente para confirmar que está de acuerdo en que se cumplió con lo requerido. Detallar. 
\end{itemize}

\end{itemize}

Tener en cuenta que en este contexto se pueden mencionar simulaciones, cálculos, revisión de hojas de datos, consulta con expertos, mediciones, etc.  

Las acciones de verificación suelen considerar al entregable como ``caja blanca'', es decir se conoce en profundidad su funcionamiento interno.  

En cambio, las acciones de validación suelen considerar al entregable como ``caja negra'', es decir, que no se conocen los detalles de su funcionamiento interno.

\end{consigna}

\section{15. Procesos de cierre}    
\label{sec:cierre}

\begin{consigna}{red}
Establecer las pautas de trabajo para realizar una reunión final de evaluación del proyecto, tal que contemple las siguientes actividades:

\begin{itemize}
	\item Pautas de trabajo que se seguirán para analizar si se respetó el Plan de Proyecto original:\\
	 - Indicar quién se ocupará de hacer esto y cuál será el procedimiento a aplicar. 
	\item Identificación de las técnicas y procedimientos útiles e inútiles que se emplearon, los problemas que surgieron y cómo se solucionaron:\\
	 - Indicar quién se ocupará de hacer esto y cuál será el procedimiento para dejar registro.
	\item Indicar quién organizará el acto de agradecimiento a todos los interesados, y en especial al equipo de trabajo y colaboradores:\\
	  - Indicar esto y quién financiará los gastos correspondientes.
\end{itemize}

\end{consigna}

\end{document}